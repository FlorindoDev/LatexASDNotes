\begin{itemize}
    \item \textbf{iostram} è libreria standard permette di formattare i caratteri in 
    \item \textbf{input} e in \textbf{output}.
    \item Le operazioni di input sono tipizzate ed estendibili cosi permettendo di gestire i tipi definiti dall'utente.
    \item Altri tipi di interazione con l’utente (grafico, ad esempio)
    non sono inclusi nello standard \textbf{ISO}(International Organization for Standardization) e quindi nemmeno in  questa libreria. \textbf{ISO} è un organizzazione di standardizzazione, per non creare anarchia.
\end{itemize}
\subsection{Output}
\begin{itemize}
    \item  Un \textbf{ostream} converte un oggetto con tipo in uno \textbf{stream} di
    caratteri (bytes).
    \item  Si può costruire l’output per ogni tipo definito dall’utente.
    \item  Operatore \textcolor{blue}{\code{<<}} (“Output”) definito per tutti gli oggetti di tipo \textbf{ostream} (es: lo standard output \code{cout}, lo standard error
    \code{cerr}, non bufferizzato o \code{clog}, bufferizzato).
    \item  Per default, i valori scritti su uno stream di uscita sono
    convertiti in una sequenza di caratteri
    \item  esempio: l’intero 10 sarà convertito nella sequenza ’1’,’0’.
    \item  Ogni operazione di \textcolor{blue}{\code{<<}} restituisce lo stream, di modo da
    poter concatenare diverse operazioni
    \lstinputlisting{Capitoli/iostream/Esempi/EsmpioOS.txt}
    \item Un carattere che viene convertito ad intero sarà convertito nella nel valore della tabella ASCII.
    \begin{tcolorbox}[width=15cm, boxsep=10pt]
        \lstinputlisting{Capitoli/iostream/Esempi/CharToInt.txt}
        \textcolor{blue}{NB:} \code{'b'} sarà convertito un ASCII quindi l'out sarà \textcolor{blue}{\code{a98c}}
    \end{tcolorbox}
\end{itemize}
\subsection{Input}
\begin{itemize}
    \item  Un \code{istream} converte uno stream di caratteri in oggetti
    con tipo.
    \item \code{istream} per input di tipi \code{built-in}, estendibile a tipi definiti
    dall’utente.
    \item Operatore \textcolor{blue}{\code{>>}} (“metto in”) definito sugli oggetti di tipo
    istream, tra cui \code{cin}.
    \item Il tipo dell’operando a destra di \textcolor{blue}{\code{>>}} determina:
    \begin{itemize}
        \item quale ingresso accetta;
        \item l’obiettivo dell’operazione.
    \end{itemize}
    \lstinputlisting{Capitoli/iostream/Esempi/EsIS.txt}
    \item Per leggere una sequenza di caratteri useremo \textcolor{blue}{\code{string}};
    tuttavia la lettura si ferma al primo carattere non
    alfanumerico:
    \lstinputlisting{Capitoli/iostream/Esempi/EsString.txt}
    \item  Per leggere un’intera linea dobbiamo usare \code{getline()}
    (vedi esempio)
    \item  Per leggere un solo carattere \code{get()}
\end{itemize}
\subsection{iostream User-Defined Types}
\begin{itemize}
    \item La libreria \code{iostream} permette di definire operazioni di I/O
    per i tipi definiti dall’utente
    \subsubsection{\textbf{\textcolor{blue}{Esmpio con Stream di output}}}
    \lstinputlisting{Capitoli/iostream/Esempi/EsOstreamUserType.txt}
    \item Nella definizione dell’operatore \textcolor{blue}{\code{<<}} per il nuovo tipo, lo
    stream di uscita
    \item viene preso, per riferimento, come primo argomento
    \item viene restituito come risultato
    \newpage
    \subsubsection{\textbf{\textcolor{blue}{Esmpio con Stream di Input}}}
    \lstinputlisting{Capitoli/iostream/Esempi/EsIS2.txt}
\end{itemize}
\newpage
\subsection{Errori iostream}
\begin{itemize}
    \item Un iostream si può trovare in uno tra quattro stati:
    \begin{itemize}
        \item \textbf{\textcolor{blue}{\code{good()}}} : la precedente operazione di \textbf{iostream} ha avuto successo\newline
        \item \textbf{\textcolor{blue}{\code{eof()}}} : arrivati alla fine dell’ingresso \textbf{(“end-of-file”)}\newline
        \item \textbf{\textcolor{blue}{\code{fail()}}} : qualcosa di inaspettato (per esempio, mi ha aspetta una cifra e ho trovato un carattere\newline
        \item \textbf{\textcolor{blue}{\code{bad()}}} : qualcosa di seriamente inaspettato (ad
        esempio, un errore nella lettura del disco)
    \end{itemize}
    \item Qualsiasi operazione venga tentata su di uno stream che
    non termina con uno stato \code{good()} non ha nessun effetto.
    \item Un iostream può venire usato come condizione: \code{true}
    solo se si trova nello stato \code{good()}
    \item Dopo un errore di lettura, per pulire lo stream e procedere:
    \lstinputlisting{Capitoli/iostream/Esempi/Esgestione errore.txt}
    \item  In alternativa, possiamo usare le eccezioni . . .    
\end{itemize}



%\lstinputlisting{Capitoli/iostream/Esempi/}