Le \textbf{eccezione} provvede a dare delle informazioni sul errore. Lo scopo delle \textbf{eccezioni} è quello di portare l’informazione dal punto in cui un errore è stato \textbf{individuato} a dove può essere \textbf{gestito}. Questa \textbf{eccezione} deve essere gestita da qualcuno. solitamente non si può gestisce nella funzione in cui viene generata(si può ma qui ci concerteremo sul non farlo). Per questo motivo la funzione lancia(\textbf{throws}) l'eccezione. Sarà la funzione chiamante ad occuparsi del eccezione.
\begin{itemize}
    \item le \textbf{librerie}: al momento della
    loro implementazione, lo sviluppatore può non conoscere
    nemmeno il tipo di programmi in cui verranno utilizzate.
    \begin{itemize}
        \item Un errore può venir individuato all’interno di una libreria,
        ma lì non si sa come trattarlo (poiché non si sa dove sarà utilizzata la libererai). quindi si lancerà un \textbf{eccezione}, cosi sarà L'Utilizzatore della libreria a gestire l'errore a meglio, in base alle sue esigenze.
        
        \item \textbf{L’utente} della libreria sa cosa fare \textbf{dell’errore}, ma non come
        individuarlo (altrimenti lo avrebbe già fatto fuori della libreria). per questo gestisce l'errore in un luogo separato perché è più semplice che identificarlo
    \end{itemize}
    \item Ovviamente questo discorso è valido di \textbf{programmi grossi} e
    strutturati la cui esecuzione si prolunga nel tempo.
    
    \item Per catturare(\textbf{catch}) queste eccezioni utilizzeremo il blocco \textcolor{blue}{\code{try-catch}} 
    \begin{itemize}
        \item Abbiamo un blocco  \textcolor{blue}{\code{try}} dove ci sono l'Istruzioni da eseguire, nel caso in cui avvenisse un errore(è stata lanciata un eccezione) ci saranno vari blocchi \textcolor{blue}{\code{catch}} per catturare L'Eccezione. Saranno catturate solo quelle specificate.
        \item invece per lanciare un eccezione si userà la parola chiave \textcolor{blue}{\code{throws}}. utilizzato dalle funzioni che non posso gestire l'errore
    \end{itemize}
    \lstinputlisting{Capitoli/Eccezioni/Esempi/EsGestExp.txt}
    \item l'eccezione Può \textbf{essere di qualsiasi tipo} che ammetta la copia, ma è raccomandabile usare un tipo definito apposta, in modo da minimizzare il \textbf{clash}(scontrarsi) tra errori lanciati da librerie diverse. (es. due eccezioni che usano il tipo interno e utilizza lo stesso numero per errori diversi)
    \item La libreria \textcolor{blue}{\code{std}} definisce una piccola gerarchia di eccezioni
    \item Un’eccezione può trasportare informazioni sull’errore che
rappresenta.
\end{itemize}
\subsection{Perché utilizzare L'Eccezione ?}
\begin{itemize}
    \item Noi utilizziamo gli L'Eccezione perché il modo tradizionale di gestione del errore non è efficiente e potrebbe portare a soluzione non corrette o mal funzionanti
    \item \textbf{\textcolor{blue}{\code{caratteristiche gestione errori tradizionale:}}}
    \begin{itemize}
        \item \textbf{\textcolor{blue}{Terminare il programma}}
        \item \textbf{\textcolor{blue}{Restituire un valore di errore}} non sempre possibile (nessun valore disponibile)
        \item \textbf{\textcolor{blue}{Stato di errore}} Restituire un valore legale, ma lasciare il programma in uno stato di errore (occorre controllare esplicitamente)
        \item Funzione che gestisce l’errore (si rimanda a come la funzione gestisce l’errore)
    \end{itemize}
     \item \textbf{\textcolor{blue}{\code{caratteristiche gestione errori con eccezioni:}}}
    \begin{itemize}
       \item \textbf{\textcolor{blue}{Codice più leggibile}} poi si può separare il codice ordinario dal quello per la gestione dei errori.
       \item  \textbf{\textcolor{blue}{il codice ordinario}} rileva l'errore e lancia l'eccezione.
    \end{itemize}
\end{itemize}
\subsection{Gerarchie degli errori della Liberia standard}
\begin{itemize}
    \item \textcolor{blue}{\code{logic\_error}} possono essere individuati o prima che cominci
    l’esecuzione o attraverso dei test sugli argomenti
    di funzioni e costruttori.
    \begin{itemize}
        \item \textcolor{blue}{\code{length\_error}}
        \item \textcolor{blue}{\code{domain\_error}}
        \item \textcolor{blue}{\code{out\_of\_range}}
        \item \textcolor{blue}{\code{invalid\_argument}}
        \item \textcolor{blue}{\code{future\_error}}
    \end{itemize}
    \item  \textcolor{blue}{\code{runtime\_error}} tutti gli altri
    \begin{itemize}
        \item \textcolor{blue}{\code{range\_error}}
        \item \textcolor{blue}{\code{overflow\_error}}
        \item \textcolor{blue}{\code{underflow\_error}}
        \item \textcolor{blue}{\code{system\_error}}
    \end{itemize}

    \item \textcolor{blue}{\code{bad\_exception}}
    \item \textcolor{blue}{\code{bad\_alloc}}
    \item \textcolor{blue}{\code{bad\_typeid}}
    \item \textcolor{blue}{\code{bad\_cast}}
\end{itemize}
\subsection{RAII(acquisizione delle risorse è l'inizializzazione)}
\begin{itemize}
    \item \code{C++} non ha un \textbf{garbage collecotr} quindi dobbiamo essere noi ad occuparci della \textbf{de-allocazione} dei oggetti allocati nel \textbf{ heap}.
    \item Quando una risorsa è \textbf{troppo grande} per lo\textbf{ stack} si alloca nel heap. avremo un \textbf{Proprietario} cioè la classe e incapsulato dentro di essa la risorsa
    \item  essendo la classe proprietario di quella risorsa deve essere lei(calsse) a deallocare la risorsa quando non ci sarà più bisogno di quella risorsa
\end{itemize}
\begin{tcolorbox}[width=15cm, boxsep=10pt]
    \lstinputlisting{Capitoli/Eccezioni/Esempi/EsRAII.txt}
    \textcolor{blue}{\textbf{NB:}} w è l'oggetto allocato nello stack, quando esce dal suo scope viene deallocato. Quando viene deallocato l'oggetto richiama il suo distruttore.
\end{tcolorbox}
\begin{itemize}
    \item Ogni risorsa viene \textbf{incapsulata} in una \textbf{classe}, in cui
    \begin{itemize}
        \item il costruttore \textbf{acquisisce} la risorsa, e lancia un’eccezione se
        non ci riesce
        \item il distruttore \textbf{rilascia} la risorsa senza mai lanciare eccezioni
        
    \end{itemize}
\end{itemize}