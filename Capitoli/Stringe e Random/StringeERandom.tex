\subsection{String}
La \textbf{standard library} mette a disposizione il tipo \code{stringa} per completare i letterali \code{stringa}. \newline il tipo stringa fornisce utili \textbf{operatori} come:

\begin{itemize}
    \item \textbf{la concatenazione(\code{+}):}
    \begin{tcolorbox}[width=15cm, boxsep=10pt]
        \lstinputlisting{Capitoli/Stringe e Random/Esempi/ConString.txt}
        \begin{itemize}
            \item  \textbf{\textcolor{blue}{NB:}} \textbf{auto} è una parola chiave che detecta il automatico il tipo che stiamo assegnato alla variabile \code{addr}.
            \item Perché non usarlo sempre ? Altrimenti cererebbe ambiguità e difficoltà nella ricerca di eventuali errori
        \end{itemize}
    \end{tcolorbox}
    \item \textbf{Operatore (\code{+=})}
    \lstinputlisting{Capitoli/Stringe e Random/Esempi/EsConstring2.txt}
\end{itemize}
Una \textbf{stringa} è \code{mutabile} quindi può cambiare nel  tempo. \newline possiamo manipolarla utilizzando anche l'operatore \textbf{\textcolor{blue}{\code{[]}}}. \newline
\begin{tcolorbox}[width=15cm, boxsep=10pt]
    \textbf{Ecco alcuni esempi:}
    \lstinputlisting{Capitoli/Stringe e Random/Esempi/EsManipolazioneStringa.txt}
\end{tcolorbox}

\begin{itemize}
    \item \textcolor{blue}{\code{sustr()}} restituisce una string che è una copia della
    stringa indicata dagli argomenti (inizio e lunghezza).
    \item \textcolor{blue}{\code{replace()}} sostituisce una sotto stringa con una stringa,
    anche di lunghezza diversa.
    \item Nei confronti tra stringhe o con letterali stringa si considera
    l’ordine lessicografico.
\end{itemize}
\subsubsection{Costruttori}
\begin{itemize}
    \item \textbf{Costruttori senza puntatori:}
    \begin{tcolorbox}[width=15cm, boxsep=10pt]
        \lstinputlisting{Capitoli/Stringe e Random/Esempi/Costr1.txt}
    \end{tcolorbox}
    \item  \textbf{Costruttore con puntatori:}
    \begin{tcolorbox}[width=15cm, boxsep=10pt]
        \lstinputlisting{Capitoli/Stringe e Random/Esempi/Costr2.txt}
    \end{tcolorbox}
    \item \textbf{allucini metodi utili:}
    \begin{tcolorbox} 
        
        \begin{itemize}
            \item \textbf{\textcolor{blue}{\code{s.size()}}}: numero di caratteri in \code{s}
            \item \textbf{\textcolor{blue}{\code{s.length()}}}: lo stesso
            \item \textbf{\textcolor{blue}{\code{s.clear()}}}: per pulire 
            \item \textbf{\textcolor{blue}{\code{s.empty()}}}: \code{bool}, è vuota?
            \item \textbf{\textcolor{blue}{\code{s.front()}}}: \code{s[0]}
            \item \textbf{\textcolor{blue}{\code{s.back()}}}: \code{s[s.size()-1]}
        \end{itemize}
    \end{tcolorbox}
\end{itemize}
\newpage
\subsection{Random}
La libreria \code{<radom>} ci permette di generare numeri \textbf{pseudo-casuali} tramite l'ausilio di due parti:
\begin{itemize}
    \item Il \textbf{motore(engine)} che genera i numeri \textbf{pesudo-casuali} secondo una distribuzione.
    \item una \textbf{distribuzione} quindi come sono distribuiti i valori, di che tipo sono e l'intervallo.  (es:
            \code{uniform\_int\_distribution}, \newline
            \code{normal\_distribution},\newline
            \code{exponential\_distribution}, . . . )\newline

    \begin{tcolorbox}[width=15cm, boxsep=10pt]
        \lstinputlisting{Capitoli/Stringe e Random/Esempi/Radomes.txt}
        \textbf{\textcolor{blue}{NB:}} Per far funzionare quesro codice c'è bisogno della libreria \code{<radom>} e \code{<functional>}
    \end{tcolorbox}
\end{itemize}
